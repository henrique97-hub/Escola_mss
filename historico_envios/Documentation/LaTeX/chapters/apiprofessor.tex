\documentclass[../main.tex]{subfiles}

\begin{document}
\section{API - Professor}

\subsection{Path}
localhost:1700/professor/(requests)

\subsection{GET}
\subsubsection{/ano/\{aluno\_id\}}
localhost:1700/professor/ano/\{aluno\_id\} \par

Recebe aluno\_id\ \par
Retorna json array de trabalhos no ano onde aluno\_id\ é igual ao pedido. \newline

\subsubsection{/ano/todos}
localhost:1700/professor/ano/todos \par

Retorna json array de trabalhos no ano de todos os alunos no db. \newline

\subsubsection{/semestre/\{aluno\_id\}}
localhost:1700/professor/semestre/\{aluno\_id\} \par

Recebe aluno\_id\ \par
Retorna json array de trabalhos, 2 arrays, 1 para cada semestre, onde aluno\_id\ é igual ao pedido \newline

\subsubsection{/semestre/todos}
localhost:1700/professor/ano/todos \par

Retorna json array de trabalhos, 2 arrays, 1 para cada semestre, de todos os alunos no db. \newline

\subsubsection{/bimestre/\{aluno\_id\}}
localhost:1700/professor/bimestre/\{aluno\_id\} \par

Recebe aluno\_id\ \par
Retorna json array de trabalhos, 6 arrays, 1 para cada bimestre, onde aluno\_id\ é igual ao pedido. \newline

\subsubsection{/bimestre/todos}
localhost:1700/professor/ano/todos \par

Retorna json array de trabalhos, 6 arrays, 1 para cada bimestre, de todos os alunos no db. \newline

\subsection{POST}

\subsubsection{/novotrabalho/\{trabalho json\}}
localhost:1700/professor/novotrabalho/\{trabalho json\} \par

Recebe json de trabalho. \par 
Cria um novo entry no db com as informações obtidas. \par
Retorna ok se der certo. \newline

\subsection{PUT}

\subsubsection{/updatetrabalho/\{trabalho\_id\}/\{trabalho json\}}
localhost:1700/professor/novotrabalho/\{trabalho\_id\}/\{trabalho json\} \par

Recebe id de trabalho e novos dados. \par 
Atualiza trabalho encontrado com o id. \par
Retorna ok se der certo. \newline



\end{document}